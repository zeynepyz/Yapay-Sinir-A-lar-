% author: Dr. Ulya Bayram
\documentclass[11pt]{article}

% Language setting
\usepackage[turkish]{babel}
\usepackage{pythonhighlight}

\usepackage[a4paper,top=2cm,bottom=2cm,left=2cm,right=2cm,marginparwidth=2cm]{geometry}

% Useful packages
\usepackage{amsmath}
\usepackage{graphicx}
\usepackage[colorlinks=true, allcolors=blue]{hyperref}
\usepackage{verbatim}
\usepackage{fancyhdr} % for header and footer
\usepackage{titlesec}
\usepackage{parskip}

\setlength{\parindent}{0pt}

\titleformat{\subsection}[runin]{\bfseries}{\thesubsection}{1em}{}

\pagestyle{fancy} % activate the custom header/footer

% define the header/footer contents
\lhead{\small{23BLM-4014 Yapay Sinir Ağları Ara Sınav Soru ve Cevap Kağıdı}}
\rhead{\small{Dr. Ulya Bayram}}
\lfoot{}
\rfoot{}

% remove header/footer on first page
\fancypagestyle{firstpage}{
  \lhead{}
  \rhead{}
  \lfoot{}
  \rfoot{\thepage}
}


\title{Çanakkale Onsekiz Mart Üniversitesi, Mühendislik Fakültesi, Bilgisayar Mühendisliği Akademik Dönem 2022-2023\\
Ders: BLM-4014 Yapay Sinir Ağları/Bahar Dönemi\\ 
ARA SINAV SORU VE CEVAP KAĞIDI\\
Dersi Veren Öğretim Elemanı: Dr. Öğretim Üyesi Ulya Bayram}
\author{%
\begin{minipage}{\textwidth}
\raggedright
Öğrenci Adı Soyadı: Zeynep Yavuz\\ % Adınızı soyadınızı ve öğrenci numaranızı noktaların yerine yazın
Öğrenci No: 190401067
\end{minipage}%
}

\date{14 Nisan 2023}

\begin{document}
\maketitle

\vspace{-.5in}
\section*{Açıklamalar:}
\begin{itemize}
    \item Vizeyi çözüp, üzerinde aynı sorular, sizin cevaplar ve sonuçlar olan versiyonunu bu formatta PDF olarak, Teams üzerinden açtığım assignment kısmına yüklemeniz gerekiyor. Bu bahsi geçen PDF'i oluşturmak için LaTeX kullandıysanız, tex dosyasının da yer aldığı Github linkini de ödevin en başına (aşağı url olarak) eklerseniz bonus 5 Puan! (Tavsiye: Overleaf)
    \item Çözümlerde ya da çözümlerin kontrolünü yapmada internetten faydalanmak, ChatGPT gibi servisleri kullanmak serbest. Fakat, herkesin çözümü kendi emeğinden oluşmak zorunda. Çözümlerinizi, cevaplarınızı aşağıda belirttiğim tarih ve saate kadar kimseyle paylaşmayınız. 
    \item Kopyayı önlemek için Github repository'lerinizin hiçbirini \textbf{14 Nisan 2023, saat 15:00'a kadar halka açık (public) yapmayınız!} (Assignment son yükleme saati 13:00 ama internet bağlantısı sorunları olabilir diye en fazla ekstra 2 saat daha vaktiniz var. \textbf{Fakat 13:00 - 15:00 arası yüklemelerden -5 puan!}
    \item Ek puan almak için sağlayacağınız tüm Github repository'lerini \textbf{en geç 15 Nisan 2023 15:00'da halka açık (public) yapmış olun linklerden puan alabilmek için!}
    \item \textbf{14 Nisan 2023, saat 15:00'dan sonra gönderilen vizeler değerlendirilmeye alınmayacak, vize notu olarak 0 (sıfır) verilecektir!} Son anda internet bağlantısı gibi sebeplerden sıfır almayı önlemek için assignment kısmından ara ara çözümlerinizi yükleyebilirsiniz yedekleme için. Verilen son tarih/saatte (14 Nisan 2023, saat 15:00) sistemdeki en son yüklü PDF geçerli olacak.
    \item Çözümlerin ve kodların size ait ve özgün olup olmadığını kontrol eden bir algoritma kullanılacaktır. Kopya çektiği belirlenen vizeler otomatikman 0 (sıfır) alacaktır. Bu nedenle çözümlerinizi ve kodlarınızı yukarıda sağladığım gün ve saatlere kadar kimseyle paylaşmayınız.
    \item Bu vizeden alınabilecek en yüksek not 100'dür. Toplam aldığınız puan 100'ü geçerse, aldığınız not 100'e sabitlenecektir.
    \item LaTeX kullanarak PDF oluşturanlar öz geçmişlerine LaTeX bildiklerini de eklemeyi unutmasınlar :)
    \item Bu vizedeki soruların çözümleri ve tex dosyası için istediğiniz kadar sayıda Github repository'si oluşturabilirsiniz. Sadece yukarıda belirttiğim tarihlerde (14 Nisan 2023 saat 15:00 ile 15 Nisan 2023 saat 15:00 arasında) public yapmayı/halka açmayı ve Github profilinizi de öz geçmişinize eklemeyi unutmayın :)
    \item Bilgisayarınıza hiçbir program kurmadan, Overleaf, Google Colab, Kaggle gibi ortamlar üzerinden bu vizeyi tamamlamanız mümkün. İyi çalışmalar!!
\end{itemize}

(Ekstra 5 puan) \url{https://github.com/zeynepyz/Yapay-Sinir-A-lar-/blob/main/overleaf.tex}

\newpage
\section{(Toplam 10 Puan) Hopfield Networks:}

\subsection{(2 Puan)} \textbf{ChatGPT’ye size Hopfield Network’ler ile ilgili detaylı bilgi vermesini isteyin. Sorduğunuz soruyu, ChatGPT'nin çıkartığı cevabı aşağı koyun. Bu cevapta anlamadığınız terimleri de ChatGPT’ye sorup, sorduğunuz soruları ve ChatGPT’nin verdiği cevapları da sırayla aşağı koyun.}

-Hopfield Network nedir bana açıklayabilir misin?
Hopfield Ağı, yapay sinir ağlarının en eski ve en yaygın kullanılan modellerinden biridir. John Hopfield tarafından 1982'de geliştirilmiştir. Bu ağ, asenkron bir güncelleme mekanizması kullanarak bir dizi girdi verisine dayalı olarak veri depolayabilir ve hatırlayabilir.

Hopfield Ağı, birçok bağlantılı nöronun etkileşimini kullanarak bir bellek işlevi gerçekleştirir. Ağ, önceden öğrenilmiş kalıpları tanıyabilir ve tanımlanmayan örüntülerin benzer öğelerini de hatırlayabilir. Bu şekilde, ağ, eksik veya bozulmuş verileri tamamlayabilir veya düzeltilebilir.

Hopfield Ağı, enerji tabanlı bir modeldir ve birçok uygulama için kullanılabilir. Örneğin, örüntü tanıma, resim işleme, sınıflandırma ve en kısa yol problemleri gibi pek çok alanda kullanılabilir.

-Enerji tabanlı bir modeli açıklayabilir misin?
Enerji tabanlı model (ETM), bir sistemin enerji fonksiyonuna dayalı olarak modellendiği bir tür yapay zeka modelidir. Bu tür modellerde, sistemdeki her durum belirli bir enerji seviyesiyle ilişkilendirilir. Daha az enerji seviyesi, sistemin daha kararlı bir durumda olduğunu gösterir.

ETM'lerde, sistemin enerjisi, değişen birçok faktörün birleşiminden oluşur. Bu faktörler, sistemdeki her bir bileşen veya öğenin durumundan kaynaklanabilir. Bu faktörler arasındaki etkileşimler, genellikle ağırlıklar veya enerji işlevleri kullanılarak temsil edilir.

ETM'ler, birçok uygulama için kullanılabilirler. Örneğin, örüntü tanıma, resim işleme, yapay zeka, robotik ve doğal dil işleme gibi pek çok alanda kullanılırlar. Bunun nedeni, ETM'lerin, özellikle de Hopfield Ağı gibi bazı ETM'lerin, birçok algoritma ve probleme uyarlanabilen esnek yapılarına sahip olmasıdır.

-Hopfield Ağı nasıl çalışır?
Hopfield Ağı, birçok bağlantılı nöronun etkileşimi ile çalışan bir yapay sinir ağıdır. Bu ağ, bir dizi girdi verisine dayalı olarak veri depolayabilir ve hatırlayabilir. Aşağıda Hopfield Ağı'nın çalışma prensipleri yer almaktadır:

Ağın başlangıç durumu, rastgele olarak veya kullanıcının tanımladığı bir girdi ile belirlenir. Bu girdi, ağın hafızasına kaydedilecektir.

Her nöron, diğer nöronlarla bağlantılıdır ve bir ağırlık matrisi tarafından belirtilir. Bu ağırlık matrisi, nöronların birbirleriyle nasıl etkileşime gireceğini belirler.

Ağın girdi verileri, ağırlık matrisindeki nöronların etkileşimleri ile bir enerji fonksiyonuna dönüştürülür. Bu enerji fonksiyonu, ağın enerji seviyesini belirler.

Ağ, enerji seviyesinin azaltılması amacıyla iterasyonel olarak çalışır. Her iterasyonda, nöronların çıktıları değiştirilir ve ağırlık matrisi güncellenir.

Ağın çıktıları, ağırlık matrisindeki nöronların etkileşimleri ile bir enerji fonksiyonuna dönüştürülür. Bu enerji fonksiyonu, ağın enerji seviyesini belirler.

Ağ, enerji seviyesinin azaltılması amacıyla iterasyonel olarak çalışır. Her iterasyonda, nöronların çıktıları değiştirilir ve ağırlık matrisi güncellenir.

Ağın enerjisi, minimum seviyeye ulaştığında, ağ sabit bir durumda kalır ve sonuçlar çıktı olarak alınabilir.

Sonuç olarak, Hopfield Ağı, enerji tabanlı bir model olarak çalışır ve ağırlık matrisindeki nöronların etkileşimleri ile bir enerji fonksiyonu kullanarak, bellek işlevlerini gerçekleştirir. Ayrıca, ağın basit yapısı ve öğrenme sürecinin kolay olması, Hopfield Ağı'nın popülerliğini artırmaktadır.









\subsection{(8 Puan)} \textbf{ChatGPT’nin açıklamalarından ne anladığınızı buraya kendi kelimelerinizle özetleyin. Örneğin ``Hopfield Network nedir? Ne işe yarar? Neden bilmemiz gerekir bunu? Gerçek hayatta kullanılıyor mu?'' gibi. Anlamadığınız kısımlar varsa ve ChatGPT’nin açıklamaları yeterli gelmezse internet dahil farklı kaynaklardan araştırıp, bilgilerin doğruluğunu sorgulamakta serbestsiniz. Konuyu doğru anlayıp anlamamanız puana etki edecektir. Kullandığınız kaynakları da belirtin!}

Hopfield Ağı, enerji tabanlı bir model ve bellek işlevleri gerçekleştirmek, eksik veya bozuk verileri tamamlamak veya düzeltmek, en kısa yol problemlerinin çözümünü yapmak için kullanılır. Bunu da girdileri hafızasına kaydederek yapıyor. Her nöron birbiriyle bağlantılı ve ağılık matrisleri tarafından belirtiliyorlar. Bu şekilde nöronlar birbiriyle etkileşim halinde oluyorlar. Girdiler bu nöronların etkileşimi sayesinde enerji fonksiyonuna dönüştürülüyor bu enerji fonksiyonu ise aslında bir hedef fonksiyonudur ve bu da ağın enerji seviyesini belirliyor. Yani bunun en küçük seviyede olmasını istiyoruz çünkü bu ağdaki düğümlerin ve bağlantılar arasındaki etkileşimin ölçüsüdür. Ve bizim de minimum değere ulaşmamız gerekiyor. Minimuma ulaştığında da ağ sabit bir durumda kalıyor ve sonuçlar çıktı olarak alınabilir hale geliyor. Bu ağı kullanmamızın sebeplerinden biri eksik veya bozuk verileri tamamlamak ya da düzeltmek. Paralel çalışma yapısından kaynaklı da büyük ölçekli problemlerde avantaj sağlıyor çünkü bu bize birçok işlemin aynı anda ve bağımsız şekilde gerçekleşmesini sunuyor. Bu şekilde işlemci kullanımını arttırıp işlem süresi kısa tutulur. Yani örneğin bir yol bulma probleminde aynı anda tüm yol maliyetlerini hesaplayabiliyor.

\section{(Toplam 10 Puan) Boltzman Machine:}

\subsection{(2 Puan)} \textbf{ChatGPT’ye size Boltzman Machine ilgili detaylı bilgi vermesini isteyin. Sorduğunuz soruyu, ChatGPT'nin çıkartığı cevabı aşağı koyun. Bu cevapta anlamadığınız terimleri de ChatGPT’ye sorup, sorduğunuz soruları ve ChatGPT’nin verdiği cevapları da sırayla aşağı koyun.}

-Boltzman Machine ilgili detaylı bilgi verebilir misin?
Boltzmann makinesi (Boltzmann Machine), bir tür yapay sinir ağıdır ve olasılık tabanlı bir modeldir. İsmi, fizikçi Ludwig Boltzmann'dan gelmektedir.

Bir Boltzmann makinesi, girdileri alır ve bu girdileri temsil eden düğümleri içerir. Bu düğümler, birbirleriyle bağlantılıdır ve her bir bağlantının ağırlığı vardır. Bu ağırlıklar, modelin öğrenmesindeki parametrelerdir. Bir Boltzmann makinesi, girdi verildiğinde, düğümler arasındaki ağırlıkların belirlediği bir olasılık dağılımını hesaplar. Bu olasılık dağılımı, verilen girdinin modele ne kadar uyduğunu gösterir.

Bir Boltzmann makinesi, birçok farklı şekilde kullanılabilir. Örneğin, bir görüntü verildiğinde, bu görüntünün ne olduğunu tahmin edebilir. Ayrıca, birçok görüntü örneği verildiğinde, bu örneklerin ortak özelliklerini öğrenebilir ve bu özellikleri kullanarak benzer görüntüleri tanımlayabilir.

Bir Boltzmann makinesi öğrenme yaparken, bir veri setindeki örnekleri kullanır ve bu örnekleri temsil eden ağırlıkları ayarlar. Bu işlem, birçok farklı öğrenme algoritması kullanılarak gerçekleştirilebilir.

Bir Boltzmann makinesi, genellikle derin öğrenme modellerinin temel bileşenlerinden biridir ve birçok farklı uygulama alanında kullanılabilir. Ancak, Boltzmann makinesi, daha gelişmiş derin öğrenme modelleri ile karşılaştırıldığında daha yavaş ve daha az verimlidir.

-Olasılık dağılımını nasıl hesaplıyor
Bir Boltzmann makinesindeki düğümler, iki farklı tip olabilir: gizli (hidden) düğümler ve görünür (visible) düğümler. Gizli düğümler, modele girdi olarak verilmeyen ve yalnızca modelin içinde bulunan düğümlerdir. Görünür düğümler ise modele girdi olarak verilen ve modele veri sağlayan düğümlerdir.

Bir Boltzmann makinesindeki her bir durum, bir enerji seviyesi ile ilişkilidir. Enerji seviyesi, görünür düğümler ve gizli düğümler arasındaki bağlantıların ağırlıklarının bir fonksiyonudur. Daha düşük enerji seviyeleri, daha olası durumları temsil ederler.

Bir Boltzmann makinesindeki her bir durum, bir olasılık ile ilişkilidir. Bu olasılık, durumun enerji seviyesine göre belirlenir. Daha düşük enerji seviyeleri, daha yüksek olasılıklara sahip durumlarla ilişkilendirilir. Bu ilişki, Boltzmann dağılımı olarak bilinir.

Bir Boltzmann makinesi, bir durumun olasılığını hesaplamak için, Boltzmann dağılımını kullanır. Bu dağılım, durumun enerji seviyesine göre belirlenen bir olasılık dağılımıdır. Bu olasılık dağılımı, girdinin modele ne kadar uyduğunu gösterir.

-Boltzmann dağılımı nedir
Boltzmann dağılımı, termodinamiğin istatistiksel mekaniği alanında kullanılan bir olasılık dağılımıdır. Bu dağılım, sistemdeki parçacıkların olası enerji düzeylerini ve bu enerji düzeylerindeki parçacıkların olasılık dağılımını tanımlar.

Bir Boltzmann makinesinde, Boltzmann dağılımı, durumun enerji seviyesine göre belirlenen bir olasılık dağılımıdır. Bu dağılım, bir Boltzmann makinesinin belirli bir durumda olma olasılığını belirler.

Boltzmann dağılımı aşağıdaki formülle ifade edilebilir:

P(E) = (1/Z)*exp(-E/kT)

Burada, P(E) enerji seviyesi E'nin olasılık dağılımını, Z bölüm fonksiyonunu, k Boltzmann sabitini ve T sıcaklığı ifade eder. Bölüm fonksiyonu, sistemin toplam enerji durumlarının sayısını temsil eder ve aynı zamanda normalleştirme sabiti olarak kullanılır.

Bir Boltzmann makinesi, bir durumun olasılığını hesaplamak için, Boltzmann dağılımını kullanır. Bu dağılım, durumun enerji seviyesine göre belirlenen bir olasılık dağılımıdır. Bu olasılık dağılımı, girdinin modele ne kadar uyduğunu gösterir.

\subsection{(8 Puan)} \textbf{ChatGPT’nin açıklamalarından ne anladığınızı buraya kendi kelimelerinizle özetleyin. Örneğin ``Boltzman Machine nedir? Ne işe yarar? Neden bilmemiz gerekir bunu? Gerçek hayatta kullanılıyor mu?'' gibi. Anlamadığınız kısımlar varsa ve ChatGPT’nin açıklamaları yeterli gelmezse internet dahil farklı kaynaklardan araştırıp, bilgilerin doğruluğunu sorgulamakta serbestsiniz. Konuyu doğru anlayıp anlamamanız puana etki edecektir. Kullandığınız kaynakları da belirtin!}

Bir Boltzmann makinesine girdi verildiğinde, düğümler arasındaki ağırlıkları belirleyip olasılık dağılımın hesaplar. Bu olasılık dağılımında ise bir değişkenin tüm olası değerleri için olasılık değerleri sağlanır ve bu değerler 1 ile 0 arasındadır. Bu makinede düğümler 2 tip olabilir: gizli ve görünür düğümler. Bu düğümler arasındaki bağlantı çift yönlüdür yani iki katman arasındaki bağlantılar hem ileriye doğru hem de geriye doğrudur anlamına geliyor. Bu demektir ki girdi düğümündeki bir nöronun çıktısı gizli düğümdeki diğer nöronların aktivasyonunu ya da gizli katmandaki bir nöronun çıktısı girdi katmanındaki diğer nöronların aktivasyonunu etkileyebilir. Bu çift yön sayesinde öğrenme ve hatırlama işlemlerini kullanabiliyor makine. Çünkü öğrenme yaparken girdi katmanındaki veriyi kullanıp gizli katmandaki nöronun ağırlığını ve eşik değerini ayarlayabilir ya da hatırlama yaparken girdi katmanındaki bir veri örneğiyle ağırlık ve eşik değerleri kullanarak gizli katman nöronlarının aktivasyonunu hesaplayabilir ve bütün bunların sonucunda çıktı katmanında tahmin yapabilir.

\section{(Toplam 10 Puan) Markov Assumption ve Markov Chain:}

\subsection{(2 Puan)} \textbf{ChatGPT’ye size Markov assumption ve Markov Chain ilgili detaylı bilgi vermesini isteyin. Sorduğunuz soruyu, ChatGPT'nin çıkartığı cevabı aşağı koyun. Bu cevapta anlamadığınız terimleri de ChatGPT’ye sorup, sorduğunuz soruları ve ChatGPT’nin verdiği cevapları da sırayla aşağı koyun.}

-Markov assumption ve Markov Chain ilgili detaylı bilgi verir misin?
Markov varsayımı, gelecekteki bir durumun sadece şu anki duruma bağlı olduğunu, önceki durumlarla ilişkisinin olmadığını varsayar. Yani, bir durumun gelecekteki olasılığı, yalnızca mevcut durumun değerine bağlıdır ve önceki durumların tarihçesi önemli değildir. Bu varsayım, birçok alanda, özellikle de doğal dil işleme ve zaman serileri analizi gibi uygulamalarda sıklıkla kullanılır.

Markov zinciri, Markov varsayımını kullanarak tanımlanan bir olasılık sürecidir. Bir Markov zinciri, bir dizi durumdan oluşur ve her bir durum, belirli bir olasılıkla bir sonraki duruma geçer. Markov zincirleri, doğal dil işleme, finans, fizik, biyoloji ve diğer birçok alanda kullanılır. Özellikle, Markov zincirleri, sistemlerin zamanla nasıl değiştiğini modellemek için kullanılır.

Markov zincirleri, birçok farklı özelliklere sahiptir. Bunlar arasında dönüşümler matrisi, dengeli dağılım, geçiş olasılıkları, erişilebilirlik ve hızlı karar verme özellikleri yer alır. Bu özellikler, bir Markov zincirinin nasıl çalıştığını anlamak için önemlidir.

Markov zincirleri, örneğin Monte Carlo simülasyonları gibi birçok farklı uygulamada kullanılır. Markov zincirleri, gelecekteki durumların öngörülmesi ve önceki durumların modellemesi için kullanıldığından, özellikle zaman serileri analizi ve doğal dil işleme alanlarında önemlidir.

-Dönüşümler matrisi, dengeli dağılım, geçiş olasılıkları, erişilebilirlik ve hızlı karar verme özelliklerini açıklar mısın?
Dönüşümler Matrisi: Markov zincirlerinin dönüşümleri matrisi, tüm olası durumların herhangi iki durum arasındaki geçiş olasılıklarının bir matris şeklinde temsilidir. Bu matrisin boyutu, zincirdeki toplam durum sayısına eşittir. Matristeki her bir eleman, bir durumdan diğerine geçiş olasılığını gösterir.

Dengeli Dağılım: Markov zinciri, sabit bir durum dağılımına yakınsar. Bu, zincirin uzun bir süre boyunca bir durumda kalma olasılığıdır. Eğer zincir dengeli bir dağılıma ulaştıysa, herhangi bir başlangıç durumundan başlamak üzere zincirdeki herhangi bir sonraki durumun olasılığı, dengeli dağılım tarafından belirlenir.

Geçiş Olasılıkları: Bir Markov zincirinde, her bir durumun bir sonraki duruma geçiş olasılığı bir dizi sayı ile ifade edilir. Bu sayılar, geçiş olasılıkları olarak adlandırılır ve zincirin dönüşümler matrisindeki ilgili elemanlarda yer alır.

Erişilebilirlik: Bir Markov zincirindeki herhangi bir durum, belirli bir olasılıkla diğer tüm durumlara geçiş yapabilir. Bu özellik, zincirin erişilebilir olduğu anlamına gelir. Erişilebilirlik özelliği, zincirin bütün durumlarına ulaşabilmeyi sağlar.

Hızlı Karar Verme: Markov zincirleri, hızlı karar verme problemlerinde kullanılabilir. Zincirin bir durumdan diğerine geçiş olasılıkları, hangi kararların verilmesi gerektiği konusunda bir model sağlayabilir. Karar verme problemlerinde, zincirin her bir durumu bir seçenek veya kararın temsil edilmesi için kullanılabilir.

-Monte Carlo simülasyonu nedir?
Monte Carlo simülasyonu, birçok farklı alanda kullanılan bir simülasyon tekniğidir. Bu teknik, bir problemi analiz etmek için rastgele sayı üretimini ve istatistiksel yöntemleri kullanır.

Monte Carlo simülasyonu, bir modelin rastgele değerler ile beslenmesi yoluyla modelin çıktısının belirlenmesiyle gerçekleştirilir. Bu rastgele değerler, modelin parametreleri, girdileri veya başka değişkenler olabilir. Rastgele değerler, birçok farklı yöntemle üretilebilir, örneğin normal dağılım, uniform dağılım veya diğer olasılık dağılımları kullanılabilir.

Monte Carlo simülasyonu, çok sayıda rastgele örnekleme yaparak, bir modelin çıktısının olası dağılımını hesaplamak için kullanılır. Bu dağılım, modelin çıktısının gerçek değeri hakkında bilgi sağlayabilir. Monte Carlo simülasyonları ayrıca, bir modelin duyarlılık analizi yapmak, riskleri tahmin etmek veya karar vermek için kullanılabilir.

Monte Carlo simülasyonları, finans, mühendislik, fizik, biyoloji ve diğer birçok alanda kullanılır. Örneğin, finansal risklerin değerlendirilmesi, bir yapısal tasarımın dayanıklılığının test edilmesi veya bir ilaç adayının etkililiğinin öngörülmesi gibi birçok uygulama için kullanılabilir.


\subsection{(8 Puan)} \textbf{ChatGPT’nin açıklamalarından ne anladığınızı buraya kendi kelimelerinizle özetleyin. Örneğin ``Markov assumption ve Markov Chain nedir? Ne işe yarar? Neden bilmemiz gerekir bunu? Gerçek hayatta kullanılıyor mu?'' gibi. Anlamadığınız kısımlar varsa ve ChatGPT’nin açıklamaları yeterli gelmezse internet dahil farklı kaynaklardan araştırıp, bilgilerin doğruluğunu sorgulamakta serbestsiniz. Konuyu doğru anlayıp anlamamanız puana etki edecektir. Kullandığınız kaynakları da belirtin!}

Markov zincirleri, Markov varsayımına dayalı olarak çalışan istatistiksel bir modellerdir. Bu zincirler bir dizi durum ve durumlar arasındaki geçiş olasılıklarından oluşuyor. Zincirin herhangi bir zamandaki durumu, önceki durum ve geçiş olasılıklarının bir fonksiyonudur. Bu zincirler bir durumda başlayıp zincirdeki durumların birbirini sırayla takip etmesiyle gerçekleşir. Bu sıra ise rastgeledir, durumlar rastgele seçilerek zincirin sonraki durumu belirlenir. Bu zincirler örneğin simülasyon modeli oluşturmada kullanılıyor. Simülasyon içinde 2 karakter hayal edelim aralarında geçecek konuşma sonrası karakterlerin tepkilerini yani sonuçları tahmin etmek için kullanılabilir. Bu gibi bir örnekte her durum bir durum olarak modellenir ve bu durumların bir sonraki durumu etkileyen belirli bir olasılığı vardır. 

\section{(Toplam 20 Puan) Feed Forward:}

\begin{itemize}
    \item Forward propagation için, input olarak şu X matrisini verin (tensöre çevirmeyi unutmayın):\\
    $X = \begin{bmatrix}
        1 & 2 & 3\\
        4 & 5 & 6
        \end{bmatrix}$
    Satırlar veriler (sample'lar), kolonlar öznitelikler (feature'lar).
    \item Bir adet hidden layer olsun ve içinde tanh aktivasyon fonksiyonu olsun
    \item Hidden layer'da 50 nöron olsun
    \item Bir adet output layer olsun, tek nöronu olsun ve içinde sigmoid aktivasyon fonksiyonu olsun
\end{itemize}

Tanh fonksiyonu:\\
$f(x) = \frac{exp(x) - exp(-x)}{exp(x) + exp(-x)}$
\vspace{.2in}

Sigmoid fonksiyonu:\\
$f(x) = \frac{1}{1 + exp(-x)}$

\vspace{.2in}
 \textbf{Pytorch kütüphanesi ile, ama kütüphanenin hazır aktivasyon fonksiyonlarını kullanmadan, formülünü verdiğim iki aktivasyon fonksiyonunun kodunu ikinci haftada yaptığımız gibi kendiniz yazarak bu yapay sinir ağını oluşturun ve aşağıdaki üç soruya cevap verin.}

\subsection{(10 Puan)} \textbf{Yukarıdaki yapay sinir ağını çalıştırmadan önce pytorch için Seed değerini 1 olarak set edin, kodu aşağıdaki kod bloğuna ve altına da sonucu yapıştırın:}

% Latex'de kod koyabilirsiniz python formatında. Aşağıdaki örnekleri silip içine kendi kodunuzu koyun
\begin{python}
import torch

x = torch.tensor([1, 2, 3], [4, 5, 6]).reshape(-1, 1)

torch.manual_seed(1) 

hidden_size = 50 
weight = torch.randn((x.shape[1], hidden_size))
bias = torch.randn((hidden_size, 1))

def sigmoid_activation(x):

  return 1 / (1 + torch.exp(-x))

def tanh(x):
    return torch.exp(x) - torch.exp(-x) / torch.exp(x) + torch.exp(-x)

hidden_output = tanh(torch.matmul(x, weight))
sonuc = sigmoid_activation(torch.matmul(hidden_output, bias))

print(sonuc)
\end{python}

tensor([[0.5349],
        [0.7648]])

\subsection{(5 Puan)} \textbf{Yukarıdaki yapay sinir ağını çalıştırmadan önce Seed değerini öğrenci numaranız olarak değiştirip, kodu aşağıdaki kod bloğuna ve altına da sonucu yapıştırın:}

\begin{python}
import torch

x = torch.tensor([[1, 2, 3], [4, 5, 6]], dtype=torch.float32).reshape(-1, 1)

torch.manual_seed(67)

hidden_size = 50
weight = torch.randn((x.shape[1], hidden_size))
bias = torch.randn((hidden_size, 1))

def sigmoid_activation(x):
    return 1 / (1 + torch.exp(-x))

def tanh_activation(x):
    return torch.tanh(x)

hidden_output = tanh_activation(torch.matmul(x, weight))
output = sigmoid_activation(torch.matmul(hidden_output, bias))

print(output)

\end{python}

tensor([[0.9868],
        [0.9406],
        [0.8711],
        [0.8340],
        [0.8284],
        [0.8385]])

\subsection{(5 Puan)} \textbf{Kodlarınızın ve sonuçlarınızın olduğu jupyter notebook'un Github repository'sindeki linkini aşağıdaki url kısmının içine yapıştırın. İlk sayfada belirttiğim gün ve saate kadar halka açık (public) olmasın:}
% size ait Github olmak zorunda, bu vize için ayrı bir github repository'si açıp notebook'u onun içine koyun. Kendine ait olmayıp da arkadaşının notebook'unun linkini paylaşanlar 0 alacak.

\url{https://github.com/zeynepyz/Yapay-Sinir-A-lar-}

\section{(Toplam 40 Puan) Multilayer Perceptron (MLP):} 
\textbf{Bu bölümdeki sorularda benim vize ile beraber paylaştığım Prensesi İyileştir (Cure The Princess) Veri Seti parçaları kullanılacak. Hikaye şöyle (soruyu çözmek için hikaye kısmını okumak zorunda değilsiniz):} 

``Bir zamanlar, çok uzaklarda bir ülkede, ağır bir hastalığa yakalanmış bir prenses yaşarmış. Ülkenin kralı ve kraliçesi onu iyileştirmek için ellerinden gelen her şeyi yapmışlar, ancak denedikleri hiçbir çare işe yaramamış.
Yerel bir grup köylü, herhangi bir hastalığı iyileştirmek için gücü olduğu söylenen bir dizi sihirli malzemeden bahsederek kral ve kraliçeye yaklaşmış. Ancak, köylüler kral ile kraliçeyi, bu malzemelerin etkilerinin patlayıcı olabileceği ve son zamanlarda yaşanan kuraklıklar nedeniyle bu malzemelerden sadece birkaçının herhangi bir zamanda bulunabileceği konusunda uyarmışlar. Ayrıca, sadece deneyimli bir simyacı bu özelliklere sahip patlayıcı ve az bulunan malzemelerin belirli bir kombinasyonunun prensesi iyileştireceğini belirleyebilecekmiş.
Kral ve kraliçe kızlarını kurtarmak için umutsuzlar, bu yüzden ülkedeki en iyi simyacıyı bulmak için yola çıkmışlar. Dağları tepeleri aşmışlar ve nihayet "Yapay Sinir Ağları Uzmanı" olarak bilinen yeni bir sihirli sanatın ustası olarak ün yapmış bir simyacı bulmuşlar.

Simyacı önce köylülerin iddialarını ve her bir malzemenin alınan miktarlarını, ayrıca iyileşmeye yol açıp açmadığını incelemiş. Simyacı biliyormuş ki bu prensesi iyileştirmek için tek bir şansı varmış ve bunu doğru yapmak zorundaymış. (Original source: \url{https://www.kaggle.com/datasets/unmoved/cure-the-princess})

(Buradan itibaren ChatGPT ve Dr. Ulya Bayram'a ait hikayenin devamı)

Simyacı, büyülü bileşenlerin farklı kombinasyonlarını analiz etmek ve denemek için günler harcamış. Sonunda birkaç denemenin ardından prensesi iyileştirecek çeşitli karışım kombinasyonları bulmuş ve bunları bir veri setinde toplamış. Daha sonra bu veri setini eğitim, validasyon ve test setleri olarak üç parçaya ayırmış ve bunun üzerinde bir yapay sinir ağı eğiterek kendi yöntemi ile prensesi iyileştirme ihtimalini hesaplamış ve ikna olunca kral ve kraliçeye haber vermiş. Heyecanlı ve umutlu olan kral ve kraliçe, simyacının prensese hazırladığı ilacı vermesine izin vermiş ve ilaç işe yaramış ve prenses hastalığından kurtulmuş.

Kral ve kraliçe, kızlarının hayatını kurtardığı için simyacıya krallıkta kalması ve çalışmalarına devam etmesi için büyük bir araştırma bütçesi ve çok sayıda GPU'su olan bir server vermiş. İyileşen prenses de kendisini iyileştiren yöntemleri öğrenmeye merak salıp, krallıktaki üniversitenin bilgisayar mühendisliği bölümüne girmiş ve mezun olur olmaz da simyacının yanında, onun araştırma grubunda çalışmaya başlamış. Uzun yıllar birlikte krallıktaki insanlara, hayvanlara ve doğaya faydalı olacak yazılımlar geliştirmişler, ve simyacı emekli olduğunda prenses hem araştırma grubunun hem de krallığın lideri olarak hayatına devam etmiş.

Prenses, kendisini iyileştiren veri setini de, gelecekte onların izinden gidecek bilgisayar mühendisi prensler ve prensesler başkalarına faydalı olabilecek yapay sinir ağları oluşturmayı öğrensinler diye halka açmış ve sınavlarda kullanılmasını salık vermiş.''

\textbf{İki hidden layer'lı bir Multilayer Perceptron (MLP) oluşturun beşinci ve altıncı haftalarda yaptığımız gibi. Hazır aktivasyon fonksiyonlarını kullanmak serbest. İlk hidden layer'da 100, ikinci hidden layer'da 50 nöron olsun. Hidden layer'larda ReLU, output layer'da sigmoid aktivasyonu olsun.}

\textbf{Output layer'da kaç nöron olacağını veri setinden bakıp bulacaksınız. Elbette bu veriye uygun Cross Entropy loss yöntemini uygulayacaksınız. Optimizasyon için Stochastic Gradient Descent yeterli. Epoch sayınızı ve learning rate'i validasyon seti üzerinde denemeler yaparak (loss'lara overfit var mı diye bakarak) kendiniz belirleyeceksiniz. Batch size'ı 16 seçebilirsiniz.}

\subsection{(10 Puan)} \textbf{Bu MLP'nin pytorch ile yazılmış class'ının kodunu aşağı kod bloğuna yapıştırın:}

\begin{python}
import torch
import torch.nn as nn

class MLP(nn.Module):
    def __init__(self, input_size, hidden_size1, hidden_size2, output_size):
        super(MLP, self).__init__()
        self.fc1 = nn.Linear(input_size, hidden_size1)
        self.fc2 = nn.Linear(hidden_size1, hidden_size2)
        self.fc3 = nn.Linear(hidden_size2, output_size)
        self.relu = nn.ReLU()
        self.sigmoid = nn.Sigmoid()
        
    def forward(self, x):
        out = self.fc1(x)
        out = self.relu(out)
        out = self.fc2(out)
        out = self.relu(out)
        out = self.fc3(out)
        out = self.sigmoid(out)
        return out

\end{python}

\subsection{(10 Puan)} \textbf{SEED=öğrenci numaranız set ettikten sonra altıncı haftada yazdığımız gibi training batch'lerinden eğitim loss'ları, validation batch'lerinden validasyon loss değerlerini hesaplayan kodu aşağıdaki kod bloğuna yapıştırın ve çıkan figürü de alta ekleyin.}

\begin{python}
kod_buraya = None
if kod_buraya:
    devam_ise_buraya = 0

print(devam_ise_buraya)
\end{python}

% Figure aşağıda comment içindeki kısımdaki gibi eklenir.
\begin{comment}
\begin{figure}[ht!]
    \centering
    \includegraphics[width=0.75\textwidth]{mypicturehere.png}
    \caption{Buraya açıklama yazın}
    \label{fig:my_pic}
\end{figure}
\end{comment}

\subsection{(10 Puan)} \textbf{SEED=öğrenci numaranız set ettikten sonra altıncı haftada ödev olarak verdiğim gibi earlystopping'deki en iyi modeli kullanarak, Prensesi İyileştir test setinden accuracy, F1, precision ve recall değerlerini hesaplayan kodu yazın ve sonucu da aşağı yapıştırın. \%80'den fazla başarı bekliyorum test setinden. Daha düşükse başarı oranınız, nerede hata yaptığınızı bulmaya çalışın. \%90'dan fazla başarı almak mümkün (ben denedim).}

\begin{python}
kod_buraya = None
if kod_buraya:
    devam_ise_buraya = 0

print(devam_ise_buraya)
\end{python}

Sonuçlar buraya

\subsection{(5 Puan)} \textbf{Tüm kodların CPU'da çalışması ne kadar sürüyor hesaplayın. Sonra to device yöntemini kullanarak modeli ve verileri GPU'ya atıp kodu bir de böyle çalıştırın ve ne kadar sürdüğünü hesaplayın. Süreleri aşağıdaki tabloya koyun. GPU için Google Colab ya da Kaggle'ı kullanabilirsiniz, iki ortam da her hafta saatlerce GPU hakkı veriyor.}

\begin{table}[ht!]
    \centering
    \caption{Buraya bir açıklama yazın}
    \begin{tabular}{c|c}
        Ortam & Süre (saniye) \\\hline
        CPU & kaç? \\
        GPU & kaç?\\
    \end{tabular}
    \label{tab:my_table}
\end{table}

\subsection{(3 Puan)} \textbf{Modelin eğitim setine overfit etmesi için elinizden geldiği kadar kodu gereken şekilde değiştirin, validasyon loss'unun açıkça yükselmeye başladığı, training ve validation loss'ları içeren figürü aşağı koyun ve overfit için yaptığınız değişiklikleri aşağı yazın. Overfit, tam bir çanak gibi olmalı ve yükselmeli. Ona göre parametrelerle oynayın.}

Cevaplar buraya

% Figür aşağı
\begin{comment}
\begin{figure}[ht!]
    \centering
    \includegraphics[width=0.75\textwidth]{mypicturehere.png}
    \caption{Buraya açıklama yazın}
    \label{fig:my_pic}
\end{figure}
\end{comment}

\subsection{(2 Puan)} \textbf{Beşinci soruya ait tüm kodların ve cevapların olduğu jupyter notebook'un Github linkini aşağıdaki url'e koyun.}

\url{www.benimgithublinkim.com}

\section{(Toplam 10 Puan)} \textbf{Bir önceki sorudaki Prensesi İyileştir problemindeki yapay sinir ağınıza seçtiğiniz herhangi iki farklı regülarizasyon yöntemi ekleyin ve aşağıdaki soruları cevaplayın.} 

\subsection{(2 puan)} \textbf{Kodlarda regülarizasyon eklediğiniz kısımları aşağı koyun:} 

\begin{python}
kod_buraya = None
if kod_buraya:
    devam_ise_buraya = 0

print(devam_ise_buraya)
\end{python}

\subsection{(2 puan)} \textbf{Test setinden yeni accuracy, F1, precision ve recall değerlerini hesaplayıp aşağı koyun:}

Sonuçlar buraya.

\subsection{(5 puan)} \textbf{Regülarizasyon yöntemi seçimlerinizin sebeplerini ve sonuçlara etkisini yorumlayın:}

Yorumlar buraya.

\subsection{(1 puan)} \textbf{Sonucun github linkini  aşağıya koyun:}

\url{www.benimgithublinkim2.com}

\end{document}
